\documentclass[../chapter01.tex]{subfile}

\begin{document}
\Section{Heurstic description with symmetric random walks}
$Y_{1}, Y_{2}, \ldots$ iid with $\Pr[Y_{i} = 1] = \Pr[Y_{i}=-1] = \frac{1}{2}$. Let $S_{k} = \sum_{i\le k}Y_{i}$ and fix $N \in \NN$. 

Give 
\begin{equation}
X_{\frac{k}{N}} = \frac{1}{\sqrt{N}}S_k
\end{equation}
Then 
\begin{enumerate}
  \item $X_0 = 0$
  \item For $0\le t_0 < t_1 < \cdots < t_m \le 1$ with $t_i = \frac{k_i}{N}$ the random variables $X_{t_i} - X_{t_{i-1}}$
    are independent and 
    \begin{equation}
      \Ex[X_{t_i} - X_{t_{i-1}}] = 0,
    \end{equation} and 
    \begin{equation}
      \Var(X_{t_i}-X_{t_{i-1}}) = \Var\left(\frac{1}{\sqrt{N}}\sum_{j={k_{i-1}+1}}^{k_{i}}Y_{j}\right) = \frac{1}{N}(k_{i}-k_{i-1}).
    \end{equation}
\end{enumerate}
Due to the CLT:\@ the laws of $X_{t_i}-X_{t_{i-1}} \xrightarrow{d}\Ncal(0, t_{i}-t_{i-1})$ for $N\to \infty$.
this motivates the definition of Brownian motion.
\end{document}
